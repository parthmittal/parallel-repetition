\documentclass[12pt]{article}

\usepackage{amsmath,amssymb, amsthm}
\usepackage{mathtools}
\usepackage{etoolbox}
\usepackage{commath}
\usepackage[svgnames]{xcolor}

% Want: On first citation, names of all authors.
\usepackage[longnamesfirst]{natbib}
\bibliographystyle{abbrvnat}

\usepackage{hyperref}
\usepackage{cleveref}

\def\eps{\varepsilon}
\def\N{\mathbb{N}}
\def\R{\mathbb{R}}

\DeclareMathOperator{\Bern}{Bern}
\DeclarePairedDelimiter\card{\lvert}{\rvert}
\DeclarePairedDelimiter\floor{\lfloor}{\rfloor}
\DeclarePairedDelimiter\paren()

\newcommand{\comment}[1]{\textcolor{red}{[\textbf{#1}]}}

\newtheorem{theorem}{Theorem}
\newtheorem{lemma}[theorem]{Lemma}
\newtheorem{claim}[theorem]{Claim}

\crefname{theorem}{theorem}{theorems}
\crefname{lemma}{lemma}{lemmas}
\crefname{claim}{claim}{claims}
\crefname{equation}{ineq.}{ineqs.}

\newcommand{\parth}[1]{\textcolor{red}{[#1]}}
\newcommand{\ort}{\textnormal{\textsc{ort}}}
\newcommand{\ghd}{\textnormal{\textsc{ghd}}}

\title{An exposition of the parallel repetition theorem}

% Begone warnings:
\hfuzz=1.0pt

\providecommand\given{}

% read the mathtools documentation about DeclarePairedDelimiterX
\DeclarePairedDelimiterX{\brac}[1][]{
\renewcommand\given{\nonscript\: \delimsize\vert{} \nonscript\: \mathopen{}}
\ifblank{#1}{\,\cdot\,}{#1}}

\DeclarePairedDelimiterX{\Set}[1]\{\}{
\renewcommand\given{\nonscript\: \delimsize\vert{} \nonscript\: \mathopen{}}
\ifblank{#1}{\,\cdot\,}{#1}}

% probability macro with support for distribution under the Pr symbol, and
% resizeable [] brackets.
% xparse arguments:
% s -> catch optional * (for auto resizing).
% O{} -> catch optional arg (for manual resizing) -- defaults to empty.
% E{_}{{}} -> catch optional argument following _ (for distribution under the
%             Pr symbol) -- defaults to empty (nb: labelled experimental)
% m -> catch mandatory argument (the event).
% examples:
% \Prob{E}, \Prob_{x}{E}, \Prob*_{x}{\frac EF}, \Prob[\Big]_{x}{E}...
% note that this also supports a resizeable conditioning symbol (via \given)
% \Prob*{E \given \frac EF}.
\NewDocumentCommand{\Prob}{sO{}E{_}{{}}m}{%
  \Pr_{#3}
  \IfBooleanTF{#1}
  {\brac*{#4}}
  {\brac[#2]{#4}}
}

\makeatletter
\def\mathcolor#1#{\@mathcolor{#1}}
\def\@mathcolor#1#2#3{%
  \protect\leavevmode
  \begingroup
  \color#1{#2}#3%
  \endgroup
}
\makeatother

\newcommand{\OA}[1]{\mathcolor{DarkRed}{#1}}
\newcommand{\OB}[1]{\mathcolor{DarkBlue}{#1}}
\newcommand{\both}[1]{\mathcolor{DarkGreen}{#1}}

\renewcommand\given{ \nonscript\: \vert{} \nonscript\: \mathopen{} }

\newcommand{\St}{\widetilde{S}}
\newcommand{\Tt}{\widetilde{T}}
\newcommand{\Ut}{\widetilde{U}}
\newcommand{\Vt}{\widetilde{V}}

\newcommand{\Xt}{\widetilde{X}}
\newcommand{\Yt}{\widetilde{Y}}

\newcommand{\Uo}{\overline{U}}
\newcommand{\uo}{\overline{u}}

\begin{document}

\maketitle

\section{The simulation}

Suppose, towards a contradiction, that there is some
$k \leq \frac{\delta^2 n}{{10}^6 \log \card{\Sigma}}$ such that:
\[
\Prob[\Bigg]{ W_j \given \bigvee_{i \leq k} W_i } \geq \delta / 100
\]
for all $j > k$.

We will try to embed an input distributed according to $P_{XY}$ into some
index $j' > k$.

For each $j > k$, define:
\begin{itemize}
  \item The uniformly random bit $D_j$, which is independent of all the other
    random variables.
  \item The random variable
    \[
      U_j = \begin{cases}
        X_j & \text{if } D_j = 0\\
        Y_j & \text{otherwise}
      \end{cases}
    \]
    and its ``complement''
    \[
      \overline{U}_j = \begin{cases}
        Y_j & \text{if } D_j = 0\\
        X_j & \text{otherwise}
      \end{cases}.
    \]
\end{itemize}

For a fixed $j' > k$ (to be decided), the protocol is the following:
\begin{enumerate}
  \item Alice receives an input $\OA{x}$ distributed according to $X$, and Bob
    receives $\OB{y}$ distributed according to $Y$.
  \item Alice computes the distribution of
    $\OA{\St\Xt_{j'} \given \Xt_{j'} = x}$,
    and Bob computes the distribution of
    $\OB{\St\Yt_{j'} \given \Yt_{j'} = y}$.
  \item Using correlated sampling, Alice and Bob sample from their respective
    distributions to obtain:
    \[
      \both{
        \underbrace{
          x_{\leq k}, y_{\leq k}, d_{> k} \setminus d_{j'},
      \uo_{> k} \setminus \uo_{j'}}_{s}},
      \OA{x_{j'}}, \OB{y_{j'}},
    \]
    distributed according to $P_{\St\Xt_{j'}\Yt_{j'}}$.
  \item Using private randomness, they both extend their part of the sample to
    obtain $\OA{x^n}$ and $\OB{y^n}$ distributed according to
    \[
      \Xt^n, \Yt^n \given \St = s, \Xt_{j'} = x, \Yt_{j'} = y.
    \]
\end{enumerate}

\subsection{From $\St, \Xt_{j'}$ to $\Xt^n$}

We would like to show that conditioning on (a fixed value of) $\St, \Xt_{j'}$,
the random variable $\Xt^n$ is independent of $\Yt^n$.
As a warm-up, we first show this without the conditioning on $W$, i.e., we
show that $X^n$ is independent of $Y^n$ after conditioning on $S, X_{j'}$.

What information does fixing $S = s$ and $X_{j'} = x_{j'}$ give us?
Straight out of the box, we know $x_{\leq k}$ and $x_{j'}$.
For any other coordinate $i \notin \brac{k} \cup \Set{j'}$:
\begin{itemize}
  \item either $d_i = 1$, in which case we know $\overline{u}_i = X_i$ from
    $S$,
  \item or $d_i = 0$, in which case we know $\uo_i = y_i$ from $s$.
    Note that $X_i$ depends only on $Y_i$, and hence conditioning on the above,
    is independent of $Y_{\leq n}$.
\end{itemize}

Returning to $\St$ and $\Xt_{j'}$, we will use additionally the fact that
a fixed value $s$ sampled from $\St$ contains a partial winning transcript
\[
  \paren{x_{\leq k}, y_{\leq k}, a_{\leq k}, b_{\leq k}}.
\]
As before for $i \in \brac{k} \cup \Set{j}$ or such that $d_i = 1$, we already
know $x_{i}$.
However, for the remaining $i$, $\Xt_i$ is not just dependent on $\Yt_i$.
For different values of $y_{> k}$, Bob could in principle output different
values $b_{\leq k}$, which (since we need to win the first $k$ games) could
change $a_{\leq k}$, and hence $x_i$.
The key observation is that fixing $S$ fixes the output of the first $k$ games,
so conditioning on this eliminates the dependence between $x_i$ and $y_{> k}$.

\end{document}
